\documentclass{article}
\usepackage[utf8]{inputenc}
\usepackage[spanish]{babel}

\begin{document}
\begin{titlepage}
    \begin{center}
        \vspace*{1cm}
            
        \Huge
        \textbf{Nociones de la memoria de un computador.}
            
        \vspace{0.5cm}
        \LARGE
        \textbf{Jose David Ortiz Miranda}
        \vfill
        \vspace{0.8cm}
        \Large
        Departamento de Ingeniería Electrónica y Telecomunicaciones\\
        Universidad de Antioquia\\
        Medellín\\
        Septiembre de 2020
            
    \end{center}
\end{titlepage}

\section{Defina que es la memoria del computador.} \label{contenido}
        \Large
        Es un espacio de trabajo donde se ejecutan múltiples tareas necesarias para el funcionamiento correcto del computador, por ejemplo, desde que se enciende el computador, en el espacio de trabajo o memoria, se ejecuta el sistema operativo que tengamos en nuestro equipo, desde que se carga el sistema operativo, el usuario puede hacer diferentes tareas en el computador, esto hace que el microprocesador y la memoria, estén siempre trabajando en conjunto para ejecutar las acciones que el usuario requiera, estas acciones son las que realizamos comúnmente en el computador, como abrir un programa y hacer cambios a un documento, o la información necesaria para ejecutar correctamente un juego, todas estas tareas se realizan en la memoria y cada tarea, ya leída por el microprocesador, es eliminada de la memoria para que esta no se llene con tareas que ya fueron ejecutadas.
        
\section{Mencione los tipos de memoria que conoce y haga una pequeña descrpcion de cada tipo.} \label{contenido}
        \Large
        Los tipos de memoria que conocía son:\\
       \textbf{Memoria RAM:}
       Es de las memorias que mas he escuchado, es importante porque es el espacio de trabajo con mas capacidad y mas importante en cualquier dispositivo y es en donde se ejecutan la mayoría de tareas que el usuario implementa.
       En esta memoria se ejecuta el sistema operativo de cualquier dispositivo, entre mas capacidad tenga, el rendimiento del dispositivo mejora notablemente.\\
       \textbf{Memoria SD:}
       Esta la implementaba mucho para los dispositivos móviles, para que los archivos no se guarden solamente en la memoria interna del dispositivo móvil, sino que se puede tener mas capacidad para almacenar información adquiriendo una de estas tarjetas de memoria que no son muy costosas.\\
       \textbf{Disco duro:}
       Lo escuchaba mucho para los computadores, es el lugar donde se almacenan permanentemente las aplicaciones o programas de un computador, y se dejan manipular por el usuario, de tal forma que pueda hacer modificaciones y trabajar con ellos.
\section{Describa la manera como se gestiona la memoria en un computador.} \label{contenido}
        \Large
        Desde que se enciende el dispositivo entran a trabajar las diferentes memorias que este tiene, la primera que empieza a trabajar es la memoria ROM, que es una memoria de solo lectura que contiene la primera instrucción para encender la maquina, esta instrucción es llamada POST, la cual hace un chequeo general de los componentes internos que contiene la maquina.
        Luego, la memoria carga un programa llamado BIOS que tiene información de los diferentes dispositivos de almacenamiento que tiene el computador, entre estos están, las tarjetas de vídeo, de sonido, la unidad de DVD, los puertos USB, entre otros, esto para que el microprocesador identifique cada uno y pueda ejecutar las instrucciones que alguno de estos dispositivos le pueda enviar.\\
        El microprocesador ya conoce la existencia del sistema operativo, y su ubicación es la memoria RAM entonces pasa a cargarlo, el sistema operativo siempre esta cargado en la memoria RAM hasta que el computador se apaga porque el microprocesador lo utilizara constantemente, entonces lo necesita en la memoria para tener acceso inmediato a el.
        Cuando el sistema operativo esta cargado ya podemos utilizar las aplicaciones que el computador tiene instaladas y se puede trabajar con ellas, de igual forma cualquier aplicación o programa que se abra en el computador, se pondrá en la memoria y cada archivo necesario para que el programa se ejecute correctamente también sera puesto en la memoria para ejecutarlo.

\section{¿Qué hace que una memoria sea más rápida que otra? ¿Por qué esto es importante?} \label{contenido}
        \Large



\end{document}