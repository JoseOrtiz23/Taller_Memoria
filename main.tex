\documentclass{article}
\usepackage[utf8]{inputenc}
\usepackage[spanish]{babel}

\begin{document}
\begin{titlepage}
    \begin{center}
        \vspace*{1cm}
            
        \Huge
        \textbf{Nociones de la memoria de un computador.}
            
        \vspace{0.5cm}
        \LARGE
        \textbf{Jose David Ortiz Miranda}
        \vfill
        \vspace{0.8cm}
        \Large
        Departamento de Ingeniería Electrónica y Telecomunicaciones\\
        Universidad de Antioquia\\
        Medellín\\
        Septiembre de 2020
            
    \end{center}
\end{titlepage}

\section{Defina que es la memoria del computador.} \label{contenido}
        \Large
        Es un espacio de trabajo donde se ejecutan múltiples tareas necesarias para el funcionamiento correcto del computador, por ejemplo, desde que se enciende el computador, en el espacio de trabajo o memoria, se ejecuta el sistema operativo que tengamos en nuestro equipo, desde que se carga el sistema operativo, el usuario puede hacer diferentes tareas en el computador, esto hace que el microprocesador y la memoria, estén siempre trabajando en conjunto para ejecutar las acciones que el usuario requiera, desde abrir un programa, hasta cerrarlo.
        
\section{Mencione los tipos de memoria que conoce y haga una pequeña descrpcion de cada tipo.} \label{contenido}
        \Large
        Los tipos de memoria que conocía son:\\
       \textbf{Memoria RAM:}
       Es de las memorias que mas he escuchado, es importante porque es el espacio de trabajo con mas capacidad y mas importante en cualquier dispositivo y es en donde se ejecutan la mayoría de tareas que el usuario implementa.\\
       \textbf{Memoria SD:}
       Esta la implementaba mucho para los dispositivos móviles, para que los archivos no se guarden solamente en la memoria interna del dispositivo, sino que se puede tener mas capacidad para almacenar información adquiriendo una de estas tarjetas de memoria que no son muy costosas.\\
       \textbf{Disco duro:}
       Lo escuchaba mucho para los computadores, es el lugar donde se almacenan permanentemente las aplicaciones o programas de un computador, y se dejan manipular por el usuario, de tal forma que pueda hacer modificaciones y trabajar con ellos.
\section{Describa la manera como se gestiona la memoria en un computador.} \label{contenido}
        \Large

\section{¿Qué hace que una memoria sea más rapida que otra? ¿Por qué esto es importante?} \label{contenido}
        \Large



\end{document}